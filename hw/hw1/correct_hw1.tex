\documentclass{article}
\usepackage{indentfirst}
\usepackage{amsmath}
\title{91.542 - Natural Language Processing ~\\ Problem Set 1}
\author{Chang Liu ~\\ chang\_liu@student.uml.edu}


\begin{document}
\maketitle


\section{Part II. Combinatorics, Probability, Information Theory}

\subsection{Problem 5. [30 pts]}

(a) \textbf{Solution:} Assume that $S_{k} = \sum_{k=1}^{K}{k*r^k}$, then $S_{k+1} = \sum_{k=1}^{K}{(k+1)*r^{k+1}}$, then we need to split the equation of $S_{k+1}$, and get the following equation:

\begin{flalign*}
S_{k+1} &= r * \sum_{k=1}^{K}{k * r^k} + r * \sum_{k=1}^{K}{r^n} \\
		&= r * S_{k} + r^2 / (1-r) % use &= to align the equation symbol
\end{flalign*}

Now I will deduction the further equation from the above one:

\begin{flalign*}
S_{k+1} &= r * S_{k} + r^2 / (1-r) \\
		&= r * (S_{k} - \frac{r^2}{(1-r)^2}) + r * \frac{r^2}{(1-r)^2} + \frac{r^2}{(1-r)} \\
		&= r * (S_{k} - \frac{r^2}{(1-r)^2}) + \frac{r^2}{(1-r)^2} \\		
\end{flalign*}

Now that from the above equation, if I move the right $r$-relate value to the left, I can then get the following equation $$S_{k+1} - \frac{r^2}{(1-r)^2} = r * (S_{k} - \frac{r^2}{(1-r)^2})$$ 

And then we can conclude that $S_{k} - r^2/(1-r)^2$ is a geometrical sequence, which meets the basic form of

\begin{equation*}
\frac{S_{k+1} - r^2/(1-r)^2}{S_{k} - r^2/(1-r)^2} = r 
\end{equation*}

Each item of the $S_{k} - \frac{r^2}{(1-r)^2}$ is a multiple of $r$ with the previous one.

According to the questions conclusion, $\sum_{n=1}^{\infty}{r^n} = \frac{r}{1-r}$, the sum of the geometrical sequence with $r^n$, the sum should be $\frac{r}{1-r}$.

Now we have another geometric sequence of $S_{k} - \frac{r^2}{(1-r)^2}$ = $[S_{1} - \frac{r^2}{(1-r)^2}]^{k-1}$. It has similar form of the $r^n$, so we can just replace the variable with similar items, using the equation to calculate the items for a geometrical sequence, we can get the general form for $S_{k}$. By combining the limitation of the $\sum_{n=1}^{\infty}{r^n} = \frac{r}{1-r}$, we can get the general item:

\begin{flalign*}
S_{n} - r^2/(1-r)^2 &= r * \frac{1}{1-r} \\
S_{n} &=  r^2/(1-r)^2 + r * \frac{1}{1-r} \\
	  &= \frac{r}{(1-r)^2}
\end{flalign*}

So we can get the value of $S_{n}$, as follows:
	
\begin{equation*}
S_{n}= \frac{r}{(1-r)^2}
\end{equation*}

~\\


\end{document}